\section{Introduction}
In a home setting, different home members may request different services through the home network. For example, someone may play games, someone may watch videos, someone may use VoIP to make voice calls and someone may use the network to download files.   
However, the bandwidth of the network in a home setting is limited. Different applications compete for relatively scarce bandwidth resoruces. But the network users demand certain levels of Quality of Service (QoS) for different services. For example, the boy in the family who loves play games would demand high and conssitent bandwidth for the games. He would like to sacrifice the bandwidth of file downloading for his games. 
Users' demands of QoS also change based on different scenarios. For example, when the boy goes to school, the parents would like to demand high QoS for watching vidoes online. However, normal users don't have the knowledge to configure the underlying network to meet their needs.

In this project, we use SDN to manage the network and provide users the flexibility to configure their QoS requirements. Users do not need to have knowledge about the underlying network, they only need to configure the QoS requirements for the services they are using. Our SDN controller will dynamically assign appropriate bandwidth to different services to satisfy a maxinum number of QoS requirements accourding to the user configuration with the limited link capacity. 

The rest of the paper proceeds as follows. 
\refsect{related} presents related work in QoS, as well as SDN-based solutions
for home and broadband access networks. 
\refsect{appraoch} describes the design of our system, as well as its implementation using Ryu and Open vSwitch.
\refsect{experiment} evaluates
our system in the
context of competing flows. \refsect{future} discusses future work
and concludes.
