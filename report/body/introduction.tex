\section{Introduction}
\label{sect:intro}

Communication networks nowadays have greatly improved the connectivity of the world. This connectivity
inspires and leads to a variety of online and real-time applications, like voice over IP (VoIP), video
streaming, online interactive gaming, video conferencing, etc. These applications impose diverse Quality
of Service (QoS) requirements on latency, bandwidth, jitter and error-rate. On the other hand, the
underlying networks can not provide unlimited bandwidth. According to~\cite{akamai}, the average
connection speed in the United States is 12.6Mbps. When users use multiple applications at the same time,
these applications compete for the relatively scarce bandwidth, thus leading to degradations of overall
performance.

A common approach to this problem is to prioritize applications' traffic flows so that it can effectively
enforce QoS of high priority applications. To this end, many QoS mechanisms have been proposed and
implemented (\todo{citations}). Nonetheless, these mechanisms have not been deployed in small scale broadband
access networks (e.g, home networks), for several reasons. First, many home routers have limited memory and
computation resources and may not be able to process and enforce complicated QoS requirements "on-the-fly"
(\todo{citations}). Second, users' demands of QoS for different applications may change in different scenarios. 
For example, in a home network setting, a user may demand high definition (HD) quality of videos while no one
is playing online video games. When other users are playing online video games, she is willing to accept standard
definition (SD) quality of videos. However, many home users do not have knowledge to configure the underlying
networks to meet their needs.

Recently Software Defined Networking (SDN) has emerged as a promising approach for providing flexible network
programmablility. SDN proposes to decouple the control plane from the data plane, and therefore it leaves the
exsiting routers and switches as simple forwarding devices. The control logic is centralized and deployed on a
server, called SDN controller, which facilitates dynamic configuration, operation and monitoring of a network.
\todo{More contents can be added here.}

In this paper, we present QoSManager, a configurable and adaptive QoS management framework based on SDN. In
QosManger, users specify high-level QoS requirements (e.g, minimum bandwidth, recommended bandwidth and priority)
for different types of traffic, and QoSManager controller will dynamically assign each flow to a queue with
appropriate rates to maximize the QoS requirements satisfied under the limited link capacity. QoSManger provides an
interface for modifying QoS requirements at runtime (configurable) and recompute the assignment of flows to queues
when flow is added to or removed from the network (adaptive).

This paper presents several contributions. First, we take into account that the QoS requirement for an application
may change in different scenarios when designing the specification language. Second, we design and implement
QoSManager, a configurable and adaptive QoS management framework using SDN support. Specifically, we propose a novel
algorithm to assign flows to queues with appropriate rates to effectively enforce QoS. Third, we demonstrate the
effectiveness of QosManager by using Mininet~\cite{mininet} and D-ITG~\cite{d-itg}.

The rest of the paper proceeds as follows. 
\refsect{related} presents related work in QoS, as well as SDN-based solutions
for home and broadband access networks. 
\refsect{appraoch} describes the design of our system, as well as its implementation using Ryu and Open vSwitch.
\refsect{experiment} evaluates
our system in the
context of competing flows. \refsect{future} discusses future work
and concludes.
