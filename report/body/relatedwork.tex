\section{Related work}
\label{sect:related}

There are various QoS techniques \todo{citations} proposed in literatures. We classify them into two main
categories: \textbf{traditional QoS strategies} which were proposed before the introduction of SDN and
\textbf{SDN enabled technologies}.

\emph{Integrated services}(IntServ)~\cite{IntServ} and \emph{Differentiated services} (DiffServ)~\cite{DiffServ}
are two main types in traditional QoS strategies. IntServ is a fine-grained and per-flow traffic control architecture.
The idea is that every network element (e.g., router, switch) in the system has to individually reserve resources for
\emph{each} flow. Thus, IntServ is not scalable. On the other hand, DiffServ is a coarse-grained traffic control
architecture. DiffServ uses a 6-bit differentiated services code point (DSCP) in the 8-bit differentiated service field
(DS field) in the IP header for packet classification purpose. DiffServ alleviates the scalability problem by providing
QoS based on aggregated flows.  In Diffserv, customers or applications request a bandwidth broker to allocate a pre-determined
amount of bandwidth. However, DiffServ is static (because of the predefined number of classes ) and lacks the ability to
fine-tune the QoS of separate flows.

In contrast to the described traditional methods, SDN enabled approaches provide more flexibility. Thanks to the abstraction
provided by the decoupled controller, users can specify policies without the need to reconfigure low-level settings at each
of the forwarding devices. Moreover, the set of policies and the number of traffic classes are unrestricted, which allows for
fine-grained tuning based on the needs of the user. Our work falls into this category and focuses on providing a easy to deploy
and configure framework for per-flow, application-based QoS management in home networks.

FlowQos~\cite{Seddiki_HotSDN14} is a framework that utilizes SDN to reconfigure home routers according to a set of rate shaping
policies defined by the user. FlowQoS and our approach share the same idea of virtual slice the network bandwidth. However,
our approach differs from FlowQoS in several ways. First, FlowQoS focuses on light weight traffic classification. To enforce
the rate shaping, FlowQoS introduces two "virtual" switches inside the home router and each virtual link between two switches 
corresponds to a different application group. After traffic classification, flows of the same application type are forwarded to
the same virtual link and the aggregated bandwidth of an application type is defined by the user. However, our approach focuses
on the assignment of an appropriate rate to each flow to maximize the QoS requirements satisfied. The bandwidth allocation of each
flow is determined dynamically and thus our framework provides more fine-grained control. In addition, we have introduced different
levels of QoS for an application type so that users have more flexibility to define their policies. Ko et al. proposed a two-tier
flow-based QoS management framework~\cite{Ko_IEICE13} that separate flow table to three different tables: flow state table, forwarding
rule table and QoS rule table where micro-flows can share a single QoS profile. Their system requires multicore processors and is
not designed for home networks. OpenQoS~\cite{Egilmez_ASPIPA12} proposed a novel controller design for QoS routing where the routes of
multimedia traffic are optimized dynamically to fulfill the required QoS. Our approach focuses on ports instead of routes and therefore
is orthogonal to their method.
