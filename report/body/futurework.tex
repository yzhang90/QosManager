\section{Conclusion and Future work}
\label{sect:future}

In this paper, we have presented QoSManager, a configurable and adaptive QoS management framework
based on SDN. Users specify high-level QoS policies through a web interface and QoSManager controller
will dynamically assign each flow to a queue with an appropriate rate. We take into account the fact
that an application may have different levels of QoS requirements in different scenarios and provide
users with great flexibility to define their policies. In the experiment, we have demonstrated that
the shared bandwidth is allocated in the way that the QoS of high priority applications are effectively
enforced.

We focused on the control module in this paper and left other modules in the framework quite simple.
The next step in our work is to improve these modules. For instance, the traffic classification module
uses a statically defined database to perform the classification. We plan to adopt machine learning
approach \todo{citation see traffic survey paper} to perform more sophisticated classification. The forwarding module now simply implements a L2
switch. We are thinking of introducing QoS based routing \cite{Apostolopoulos_SIGCOMM98, Egilmez_ASPIPA12}
into the module. In addition, we are planning to provide more flexibility for users to define time-based
QoS and device-based QoS.
